% Options for packages loaded elsewhere
\PassOptionsToPackage{unicode}{hyperref}
\PassOptionsToPackage{hyphens}{url}
%
\documentclass[
]{article}
\usepackage{amsmath,amssymb}
\usepackage{iftex}
\ifPDFTeX
  \usepackage[T1]{fontenc}
  \usepackage[utf8]{inputenc}
  \usepackage{textcomp} % provide euro and other symbols
\else % if luatex or xetex
  \usepackage{unicode-math} % this also loads fontspec
  \defaultfontfeatures{Scale=MatchLowercase}
  \defaultfontfeatures[\rmfamily]{Ligatures=TeX,Scale=1}
\fi
\usepackage{lmodern}
\ifPDFTeX\else
  % xetex/luatex font selection
\fi
% Use upquote if available, for straight quotes in verbatim environments
\IfFileExists{upquote.sty}{\usepackage{upquote}}{}
\IfFileExists{microtype.sty}{% use microtype if available
  \usepackage[]{microtype}
  \UseMicrotypeSet[protrusion]{basicmath} % disable protrusion for tt fonts
}{}
\makeatletter
\@ifundefined{KOMAClassName}{% if non-KOMA class
  \IfFileExists{parskip.sty}{%
    \usepackage{parskip}
  }{% else
    \setlength{\parindent}{0pt}
    \setlength{\parskip}{6pt plus 2pt minus 1pt}}
}{% if KOMA class
  \KOMAoptions{parskip=half}}
\makeatother
\usepackage{xcolor}
\usepackage[margin=1in]{geometry}
\usepackage{graphicx}
\makeatletter
\def\maxwidth{\ifdim\Gin@nat@width>\linewidth\linewidth\else\Gin@nat@width\fi}
\def\maxheight{\ifdim\Gin@nat@height>\textheight\textheight\else\Gin@nat@height\fi}
\makeatother
% Scale images if necessary, so that they will not overflow the page
% margins by default, and it is still possible to overwrite the defaults
% using explicit options in \includegraphics[width, height, ...]{}
\setkeys{Gin}{width=\maxwidth,height=\maxheight,keepaspectratio}
% Set default figure placement to htbp
\makeatletter
\def\fps@figure{htbp}
\makeatother
\setlength{\emergencystretch}{3em} % prevent overfull lines
\providecommand{\tightlist}{%
  \setlength{\itemsep}{0pt}\setlength{\parskip}{0pt}}
\setcounter{secnumdepth}{-\maxdimen} % remove section numbering
\ifLuaTeX
  \usepackage{selnolig}  % disable illegal ligatures
\fi
\usepackage{bookmark}
\IfFileExists{xurl.sty}{\usepackage{xurl}}{} % add URL line breaks if available
\urlstyle{same}
\hypersetup{
  pdftitle={Water and Sanitation Deprivation in Kenya},
  pdfauthor={Cynthia Anguza},
  hidelinks,
  pdfcreator={LaTeX via pandoc}}

\title{Water and Sanitation Deprivation in Kenya}
\author{Cynthia Anguza}
\date{2024-10-28}

\begin{document}
\maketitle

\subsubsection{Introduction}\label{introduction}

This R Shiny application visualizes water and sanitation deprivation
data in Kenya. It provides interactive maps and plots to explore changes
over time in water sources and sanitation access between 2009 and 2019.

\subsubsection{Libraries Used}\label{libraries-used}

\begin{itemize}
\tightlist
\item
  \textbf{shiny}: Framework for building interactive web applications in
  R.
\item
  \textbf{shinythemes}: Provides themes for Shiny applications.
\item
  \textbf{leaflet}: Used for creating interactive maps.
\item
  \textbf{plotly}: Enables interactive plots.
\item
  \textbf{tidyverse}: A collection of R packages for data manipulation
  and visualization.
\item
  \textbf{sf}: Used for handling spatial data.
\end{itemize}

\subsubsection{2. Setting Working Directory and Loading
Data}\label{setting-working-directory-and-loading-data}

\section{Set working directory}\label{set-working-directory}

setwd(``E:/Learning'')

\section{Load shapefiles}\label{load-shapefiles}

kenya\_shp\_adm1 \textless-
st\_read(``ken\_admbnda\_adm1\_iebc\_20191031.shp'')

\section{Load the CSV data}\label{load-the-csv-data}

data \textless-
read.csv(``E:/Learning/water\_sanitation\_2009\_2019.csv'')

\subsubsection{3. Data Diagnostics}\label{data-diagnostics}

\section{Print the structure of the data for
diagnostics}\label{print-the-structure-of-the-data-for-diagnostics}

print(str(data))

\section{Ensure column names are
correct}\label{ensure-column-names-are-correct}

colnames(data) \textless- gsub('' ``,''.'', colnames(data))
print(colnames(data)) \# Print column names for verification

\subsubsection{4. Data Merging}\label{data-merging}

\section{Merge the shapefile data with the CSV data by
county}\label{merge-the-shapefile-data-with-the-csv-data-by-county}

kenya\_shp\_adm1 \textless- kenya\_shp\_adm1 \%\textgreater\%
left\_join(data, by = c(``ADM1\_EN'' = ``County''))

\subsubsection{5. User Interface}\label{user-interface}

The UI includes a title, custom CSS for styling, and multiple tabs:

\begin{itemize}
\tightlist
\item
  \textbf{Home}: Provides a brief overview of water and sanitation
  deprivation.
\item
  \textbf{Map View}: Displays an interactive map of Kenya showing water
  access in 2019 by county.
\item
  \textbf{Data Visualization}: Contains graphs illustrating water and
  sanitation indicators over time.
\end{itemize}

\section{Define UI for the
application}\label{define-ui-for-the-application}

ui \textless- fluidPage( theme = shinytheme(``flatly''),

\# Custom CSS for background image tags\(head(
    tags\)style(HTML('' \#home-content \{ position: relative;
background-image:
url(`\url{https://images.pexels.com/photos/1446504/pexels-photo-1446504.jpeg?auto=compress&cs=tinysrgb&w=600}');
background-size: cover; background-position: center; color: white;
padding: 50px; \} \#home-overlay \{ position: absolute; top: 0; left: 0;
right: 0; bottom: 0; background: rgba(0, 0, 0, 0.6); /* Semi-transparent
black \emph{/ z-index: 1; \} \#home-content h3, \#home-content p \{
position: relative; z-index: 2; /} Bring text above the overlay \emph{/
background: rgba(0, 0, 0, 0.4); padding: 10px; border-radius: 5px; \} h3
\{ font-size: 2em; /} Increase size for headers \emph{/ \} p \{
font-size: 1.2em; /} Increase size for paragraphs */ \} ``)) ),

titlePanel(``Water and Sanitation Deprivation in Kenya''),

tabsetPanel( tabPanel(``Home'', div(id = ``home-content'', div(id =
``home-overlay''), h3(``Overview of Water and Sanitation Deprivation''),
p(``Nearly 4 in 10 Kenyans did not have access to safe drinking water in
2019 and 2 in 10 were deprived of adequate sanitation. However, there
were remarkable improvements in both sectors between 2009 and 2019,
albeit the decrease in deprivation was stronger in sanitation where it
almost halved. Safe drinking water and basic sanitation are essential
for the survival of children.''), tags\$div( style = ``margin-top: 20px;
line-height: 1.5;'', HTML(''

Strong efforts need to be put to tackle issues in the sector, as Kenya
is classified as a water-scarce country. This, coupled with more
frequent cycles of severe and unpredictable weather conditions and
increased rates of natural resource depletion, will make water less
available, especially in the country's arid and semi-arid areas, calling
for urgent and sustainable solutions.

\begin{verbatim}
                 <p>Another structural issue facing the sector is that the water service providers in Kenya struggle 
                 to raise the capital and strengthen local capacities needed to accelerate water delivery. 
                 Inequalities in access to water and sanitation were large across areas of residence and counties. 
                 Rainfall patterns as well as existing investments by national and county government, as well as 
                 international partners, are some of the key factors that explain part of these differences.</p>
                 <p>In 2019, the share of the population in rural areas deprived in water was more than twice that in urban 
                 areas, 46 versus 21 percent, respectively. Likewise, while nearly 3 in 10 persons in rural areas 
                 were deprived of adequate sanitation, in urban areas the deprivation rate was 1 in 10 persons.</p>
               ")
             )
         )
),

tabPanel("Map View",
         leafletOutput("kenya_map", height = 600)
),

tabPanel("Data Visualization",
         mainPanel(
           tabsetPanel(
             tabPanel("Water Source Graph",
                      plotlyOutput("water_graph", height = 600, width = 1200)
             ),
             tabPanel("Sanitation Graph",
                      plotlyOutput("sanitation_graph", height = 600, width = 1200)
             ),
             tabPanel("Changes",
                      selectInput("change_indicator", "Select Change Indicator", 
                                  choices = c("Change in Water", "Change in Sanitation")),
                      plotlyOutput("changes_graph", height = 600, width = 1200)
             )
           )
         )
)
\end{verbatim}

) )

\subsubsection{6. Server}\label{server}

The server function contains the logic for rendering the map and plots
based on user input and data. It includes three outputs: a map view with
counties colored by water source data, and two graphs to visualize
changes in water and sanitation data over time.

\section{Define server logic}\label{define-server-logic}

server \textless- function(input, output, session) \{

\# Map output output\(kenya_map <- renderLeaflet({
    pal_water <- colorQuantile("YlGnBu", kenya_shp_adm1\)Water.Source.2019,
n = 5)

\begin{verbatim}
leaflet(kenya_shp_adm1) %>%
  addTiles() %>%
  addPolygons(
    fillColor = ~pal_water(Water.Source.2019),
    weight = 2,
    opacity = 1,
    color = "white",
    dashArray = "3",
    fillOpacity = 0.7,
    highlight = highlightOptions(
      weight = 5,
      color = "#666",
      dashArray = "",
      fillOpacity = 0.7,
      bringToFront = TRUE
    ),
    popup = ~paste(
      "County: ", ADM1_EN,
      "<br>Water Source 2009: ", Water.Source.2009,
      "<br>Water Source 2019: ", Water.Source.2019,
      "<br>% Change Water: ", X..Change.Water.Source,
      "<br>Sanitation 2009: ", Sanitation.2009,
      "<br>Sanitation 2019: ", Sanitation.2019,
      "<br>% Change Sanitation: ", X..Change.Sanitation
    )
  ) %>%
  addLegend(pal = pal_water, values = ~Water.Source.2019, opacity = 0.7, title = "Water Supply", position = "bottomright")
\end{verbatim}

\})

\# Water Graph output output\$water\_graph \textless- renderPlotly(\{
plot\_data \textless- data \%\textgreater\% select(County,
Water.Source.2009, Water.Source.2019) \%\textgreater\%
pivot\_longer(cols = c(Water.Source.2009, Water.Source.2019), names\_to
= ``Year'', values\_to = ``Value'')

\begin{verbatim}
plot_ly(plot_data, x = ~County, y = ~Value, color = ~Year, type = 'scatter', mode = 'lines+markers',
        line = list(width = 2),
        marker = list(size = 6),
        colors = c("Water.Source.2009" = "red", "Water.Source.2019" = "green")) %>%
  layout(title = "Water Indicators Over Time",
         xaxis = list(title = "County", tickangle = -45, automargin = TRUE, tickmode = "array", tickvals = plot_data$County, tickfont = list(size = 10)),
         yaxis = list(title = "Water Source (%)", titlefont = list(size = 16)),
         margin = list(b = 300, t = 50, l = 80, r = 50),
         width = 1200,  
         showlegend = TRUE) %>%
  config(displayModeBar = FALSE)
\end{verbatim}

\})

\# Sanitation Graph output output\$sanitation\_graph \textless-
renderPlotly(\{ plot\_data \textless- data \%\textgreater\%
select(County, Sanitation.2009, Sanitation.2019) \%\textgreater\%
pivot\_longer(cols = c(Sanitation.2009, Sanitation.2019), names\_to =
``Year'', values\_to = ``Value'')

\begin{verbatim}
plot_ly(plot_data, x = ~County, y = ~Value, color = ~Year, type = 'scatter', mode = 'lines+markers',
        line = list(width = 2),
        marker = list(size = 6),
        colors = c("Sanitation.2009" = "blue", "Sanitation.2019" = "green")) %>%
  layout(title = "Sanitation Indicators Over Time",
         xaxis = list(title = "County", tickangle = -45, automargin = TRUE, tickmode = "array", tickvals = plot_data$County, tickfont = list(size = 10)),
         yaxis = list(title = "Sanitation (%)", titlefont = list(size = 16)),
         margin = list(b = 300, t = 50, l = 80, r = 50),
         width = 1200,
         showlegend = TRUE) %>%
  config(displayModeBar = FALSE)
\end{verbatim}

\})

\# Changes Graph output output\(changes_graph <- renderPlotly({
    if (input\)change\_indicator == ``Change in Water'') \{ plot\_data
\textless- data \%\textgreater\% select(County, X..Change.Water.Source)
title\_text \textless- ``\% Change in Water (2009-2019)'' \} else \{
plot\_data \textless- data \%\textgreater\% select(County,
X..Change.Sanitation) title\_text \textless- ``\% Change in Sanitation
(2009-2019)'' \}

\begin{verbatim}
plot_ly(plot_data, x = ~County, y = ~get(input$change_indicator), type = 'bar', marker = list(color = 'lightblue')) %>%
  layout(title = title_text,
         xaxis = list(title = "County", tickangle = -45, automargin = TRUE, tickmode = "array", tickvals = plot_data$County, tickfont = list(size = 10)),
         yaxis = list(title = "% Change", titlefont = list(size = 16)),
         margin = list(b = 300, t = 50, l = 80, r = 50),
         width = 1200)
\end{verbatim}

\}) \}

\subsubsection{7. Running the App}\label{running-the-app}

To run the Shiny app, execute the following code in R:
shiny::runApp(``E:/Learning/water and sanitation.R'')

\end{document}
